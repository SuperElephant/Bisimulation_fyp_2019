\section{Introduction}
% This will give a brief overview of the project including

% What problem is addressed by the project?
% What are the aims and objectives of the project?
% What are the challenges of the project?
% What is the solution produced?
% How effective is the solution / how successful has the project been?

Artificial Neural Networks (ANNs) has been widely used in multiple areas. 
It has been proved that ANN is Turing complete \cite{siegelmann1995computational}. 
It indicates that understanding the logic (i.e. symbolic computation) is possible theoretically, gives broad application prospects.
There are ANNs \cite{holldobler1999approximating, Holldobler91towardsa} which are able to mimic some first-order logic program. 
However, these solutions were based on the Recurrent Neural Network (RNN) and tried to simulate the logic program rather than catch the logic feature. 
Trying to explore the limitation of ANNs for understanding logical concepts, will help the research of neuro-symbolic integration.

Thus, here we are concerned with the capability of different  Multilayer Perceptrons (MLPs) to understand the logic concept (i.e. bisimulation of logic structures ). 
And MLPs here is introduced as the representative of ANNs, as it is recognised as the original and most basic model of ANNs.
Datasets used to train the MLPs is consisted of groups of generated graphs with flags (i.e. a number that indicates if two graphs are bisimulation).
Then test if MLPs are able to learn the ability to distinguish the bisimulation equivalence.
The report is structured as follows.
First, in Section \ref{sec:background}, relevant research of bisimulation and relation between neural network and logic will be given. 
The specification of the problem and the approach used to explore the limitation will be explained in Section \ref{sec:description}.
Then in Section \ref{sec:relisation}, the process of realisation including component test and the process of the project will be stated.
Section \ref{sec:experiment} will discuss the entire process of the experiments including the setting up, impelementation and analysation of result.
And Section \ref{sec:evaluation}, the summary and the evaluation from the aspect of engineer and research will be explained.
Knowledge and skills gained will be talked in Section \ref{sec:learning}.
The professional issue will be discussed in Section \ref{sec:profession}.
At last the full code of the project, part of the data, screenshots of sample runs, user manual, progress log and Gantt chart will be appended in the Section \ref{sec:appendics}.