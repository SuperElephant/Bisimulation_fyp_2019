\section{Learning Points}
\label{sec:learning}
% At least one page of summary of the key learning points in the project.
In this section all the materials and skill had been learning will be listed.
During the development of this project, many things had been learned, including the theories, programming skills and research skills.
\begin{itemize}
    \item \textbf{Theories}
    \begin{itemize}
        \item Bisimulation 
        \begin{itemize}
            \item Online Course: System Validation: Automata \& behavioural equivalences \cite{Groote2016}
            \item Thesis reading, about modal logic (e.g. \cite{VanBenthem1976,Blackburn2001a}), model checking (e.g. \cite{Roscoe1994}), bisimulation (e.g \cite{Glabbeek2011, Stirling2011a, Larsen1991}) and relevant algorithm (e.g. \cite{Paige1987, Dovier2004})
            \item Understand the bisimulation and the algorithm
        \end{itemize}
        
        \item Machine learning
        \begin{itemize}
            \item Online Course: Machine Learning by Andrew Ng \cite{Ng2015}
            \item Thesis reading, about neural network (e.g. \cite{Dayhoff1990}), machine learning on logic (e.g. \cite{holldobler1999approximating, Leshno2013})
        \end{itemize}
        
    \end{itemize}
    
    \item \textbf{Programming Skills}
    \begin{itemize}
        \item programming in \texttt{Python} with \texttt{PyCharm}
        \item Using \texttt{TensorFlow} to construct a neural network with visualisation, i.e. \texttt{TensorBoard}
        \item Using GPU to train the neural network
        \item Using package like \texttt{matplotlib, graphviz} to visualise the graphs
        \item Using \texttt{Git} and \texttt{Github} to manage the project
        \item Using remote computing server i.e. Google Cloud Platform
        \item Deploying experiments on \texttt{Linux}
        \item Using \LaTeX to typeset the dissertation
    \end{itemize}
    \item \textbf{Research Skills}
    \begin{itemize}
        \item Searching relevant thesis
        \item Reading, understanding and reproducing the algorithm of the thesis especially \cite{Paige1987}.
        \item Scheduling the project with Gantt-chat
        \item Analysis of information from different sources
    \end{itemize}
\end{itemize}
Nevertheless, except the skills had been learned.
There are also many skills that need to be improved or mastered.
The most important one is the skill to find the information.
At the first stage of the project, the research focus on the bisimulation. 
Meanwhile, the literature about the neural network on logical learning was not collected and reviewed sufficiently, which affected later study on machine learning.
The literature review should cover the whole project before actually working on it, rather than reviewing partially.

